\chapter[Tecnologías]{Tecnologías}
\label{Chap3}

Para este proyecto, es necesario el uso de dos maquinas virtuales para crear los servidores necesarios, un servidor de base de datos y otro para el despliegue de una API y la obtencion de datos. Aunque hacer uso de una única máquina no solo es viable, si no más sencillo, disponer de ellas, no solo aparta la plataforma de un diseño centralizado más vulnerable, ademas, a nivel de proyecto, proporciona un mayor grado de complejidad, necesitando configurar la comunicación por red de estas.\newline
\newline
Primero que todo es necesario un sistema operativo capaz de servir este propósito.

\section{Sistema Operativo Debian}
En 1993, tras varios intento fallidos por distintas empresas de solucionar el problema, Ian Murdock, por aquel entonces estudiante de la Universidad Purdue, encontró la solución al problema basándose en el reciente proyecto de Linus Torvalds, el kernel Linux. tras el anuncio de Ian para crear un sistema operativo de forma descentralizada en paralelo como es el caso del kernel Linux, docenas de usuarios se unieron para formar el proyecto Debian Linux con la intención de crear un sistema operativo de gran calidad y mantenimiento, publicando en enero de 1994 la primera versión de Debian 0.91 \cite{krafft2005debian} \cite{DebHis}.\newline
\newline
Debian no intenta seguir ni competir con los lideres del sector, por el contrario, desde sus inicios el proyecto se ha basado en una filosofía centrada en la robustez y estabilidad del sistema guiada por unos estrictos estándares de calidad, actualizándose conforme las necesidades de sus usuarios a la vez que promueve el software gratuito, lo que le ha ayudado a obtener fama entre los usuarios \cite{DebFil} \cite{pollei2013debian}.\newline
\newline
A su vez, debido al apoyo del software libre, hace uso de múltiples licencias de software como la Licencia Pública General GNU (GPL), licencias artísticas o del tipo BSD, lo cual ha llevado al desarrollo de las Directrices de Software Libre de Debian (DFSG) con el fin de definir la construcción del software libre \cite{DebFree}.\newline
\newline
Definido por estas licencias, el software libre debe cumplir al menos las que están consideradas la cuatro libertades esenciales \cite{GnuFS}:

\begin{itemize}
	\item Ejecutar el programa como se desee, con cualquier propósito.
	\item Estudiar cómo funciona el programa, y cambiarlo para que haga lo que se desee.
	\item Redistribuir copias para ayudar a otros.
	\item Distribuir copias de sus versiones modificadas a terceros permitiendo ofrecer a toda la comunidad la oportunidad de beneficiarse de las modificaciones.
\end{itemize}

El mercado está repleto de distintas posibles alternativas a Debian, ya sean de pago, Windows Server OS o Red Hat Enterprise Linux (RHEL), gratuitos, Ubuntu Server y Fedora Server, o incluso en la nube, Amazon Web Services (AWS), Google’s Cloud Platform.
\newline
\newline
Descartando todo sistema de pago, aunque las posibilidades se reducen, aun hay múltiples opciones sobre las que elegir, pero son pocas las que ofrecen la misma usabilidad que Debian con sus más de 59000 paquetes en su versión estable \cite{DebWhy}.
\newline
\newline
Aunque todos los sistemas basados en Linux disponen de las mismas características, siendo software libre y gratuito con soporte multi-usuario, multi-proceso y uso en tiempo real, Debian siendo uno de los sistemas más longevos del mercado a tomado fama entre la competencia por su seguridad y estabilidad. Siendo la base para muchas de las distribuciones más populares contra las que compite, como Ubuntu, Knoppix, PureOS o Tails.
\newline
\newline
Cabe mencionar, que parte de esta seguridad y estabilidad puede llegar a ser un limitante a la hora de elegir Debian como sistema operativo dependiendo de tus necesidades a nivel de uso pues el software compatible igual no es la versión más reciente.
\newline
\newline
Finalmente, una característica distintiva de Debian frente a la competencia gratuita es su compatibilidad con un uso 24/7, pues otras alternativas gratuitas o no dan soporte a esta característica, Fedora Server o, necesitas disponer de una licencia de pago como es el caso de Ubuntu Server mediante Ubuntu Pro.
\newline
\newline
Puesto que no afecta negativamente al proyecto no disponer de las versiones más recientes de software y debido a la necesidad de un sistema 24/7 gratuito, parece lógica la elección de Debian como sistema operativo para usar en el servidor.\newline
\newline
Una vez seleccionado sistema operativo, hace falta selecionar un Framework para crear y trabajar con una API.

\section{Framework necesarios}
Un framework es software que provee una infraestructura básica sobre la que desarrollar tus proyectos. Aportan las funcionalidades y estructuras básicas necesarias para este sin la necesidad de programar todo desde cero, ahorrando tiempo de desarrollo y aportando robustez al proyecto \cite{ghimire2020comparative}.
\newline
\newline
Cada framework aportará su propia colección de módulos y paquetes específicos para ayudar en el desarrollo, es por esto que generalmente se clasifican en tres clases distintas según funcionalidades: Full Stack, Micro y Asíncrono \cite{WebFra}.\newline
\newline
Full-stack es un tipo de framework apto tanto para deserrollo back-end como front-end, aportando todas las herramientas posibles que ayuden con el desarrollo gráfico de la interfaz de usuario (UI), gestión de bases de datos, protocolos de seguridad y lógica de negocio entre tantos. Siendo Django un ejemplo de framework full-stack. 
\newline
\newline
Micro es un tipo de framework ligero por definición, siendo en cierta medida lo contrario de un framework full-stack, pues aunque los componentes que aportan como puede ser la gestión de bases de datos son los mismos, estos no vienen incluidos de forma nativa. Esto se debe a que buscan aportar flexibilidad y libertad a los desarrolladores para que incluyan unicamente aquellas herramientas que necesiten. Como se explica en la documentación de Flask, uno de los framework tipo micro más relevante, el 'micro' de microframework significa que el núcleo del framework es simple pero extensible.
\newline
\newline
Asíncrono es un tipo de framework dirigido por eventos. En vez de hacer un manejo operacional linea a linea de las funciones en la que se van ejecutando una detrás de otra, el código asíncrono no es bloqueante por lo que no se espera que un evento termine para ejecutar el siguiente, ejecutándolo de forma simultanea. Debido a esto un framework asíncrono puede llegar a conseguir un gran rendimiento si se usa en un servidor que lo permita.

\subsection{Librería vs Framework}
Aunque ambas ofrecen funcionalidades operacionales, su mayor diferencia radica en la especificidad y complejidad de estas. Las librerías están compuestas por múltiples métodos para un uso especifico sin aportar mucha complejidad, realizando una tarea por función.
\newline
\newline
Por el contrario, los framework tienen en cuenta las posibles necesidades de tu proyecto, pudiéndose permitir ser aún más específicos, ofreciendo la arquitectura y comportamiento básico de la aplicación, sin comprometer la flexibilidad de desarrollar las funcionalidades necesarias para su funcionamiento, aportando herramientas sobre las que trabajar \cite{glez2014web}.

\subsection{Django Framework}
Djanfo es un framework web de tipo full-stack gratuito y de código abierto para Python. Sigue el principio DRY "Don’t Repeat Yourself" por lo que se enfoca en el menor uso de código, el desarrollo rápido y la reutilización de componentes \cite{DjDF} \cite{alchin2013pro}.
\newline
\newline
Hace uso de su auto denominado patrón Modelo-Vista-Template (MVT), mostrado en la figura \ref{fig:ej1}, una variante del conocido Modelo-Vista-Controlador (MVC) \cite{ravindran2015django}.

\begin{figure} [H]
	\centering
	\includegraphics[width=0.7\textwidth]{fig/DjangoMVT.png}
	\caption[Diagrama patrón MVT]{Diagrama patrón MVT \footnotemark}
	\label{fig:ej1}
\end{figure}
\footnotetext{https://docs.hektorprofe.net/django/web-personal/patron-mvt-modelo-vista-template/}

Para poder trabajar con bases de datos relacionales tales como, Oracle, MySQL y PostgreSQL, Django usa un Mapeador Relacional de Objetos (ORM) que permite interactuar con ellas mediante SQL.\newline
\newline
Django es uno de los frameworks más reputados en Python, teniendo en cuanta su naturaleza gratuita y de código abierto. Siendo usado tanto por la comunidad como por empresas tales como Instagram, Mozilla y Facebook. Aunque no por ello significa que no disponga de poca competencia, siendo TurboGears, web2py, Bottle y Flask de las más notables.
\newline
\newline
El que sea un framework full-stack no va a aportar la flexibilidad, libertad y ligereza que da el uso de un microframework, sobre todo a la hora de agregar únicamente aquellas herramientas que considere necesarias, pero si que atenuará la carga de trabajo que puede suponer un microframework si no se dispone de la experiencia necesaria para usarlo.
\newline
\newline
Otra característica por la que decantarse por Django, aunque no sea única de él, es su compatibilidad con bases de datos relacionales de forma nativa, cosa que facilitará el uso de estas en el proyecto.\newline
\newline
A su vez, con el fin de obtener los datos, se necesita de una herramienta centrada en el scrapeo de datos.

\section{Herramienta para Web Scraping}
Web scraping, también conocido como web extraction o web harvesting, es una técnica de extracción de datos desestructurados de la World Wide web (WWW) y guardarlos de forma estructurada en una base de datos o en un sistema de ficheros en formato XML, JSON o CSV para su posterior recuperación o análisis. Generalmente, los datos web son adquiridos mediante el uso de Hyper-text Transfer Protocol (HTTP) o a través de un navegador web, ya sea de forma manual o automática mediante web crawlers, herramientas diseñadas con este propósito, siendo capaces de convertir páginas web enteras en información bien estructurada \cite{zhao2017web} \cite{krotov2018legality}.
\newline
\newline
Debido a la gran cantidad de datos que son generados constantemente en la WWW, web scraping es considerado una forma eficiente y poderosa de amasar big data. Pudiendose usar en una gran variedad de entornos, como la recolección de comentarios en redes sociales, listado de la propiedad inmobiliaria o en este caso el monitoreo y comparación de los niveles de los ríos y datos pluviométricos.

\subsection{Procedimiento básico en Web Scraping}
El proceso de recolección de datos de Internet se puede dividir en tres procesos secuenciales, el estudio de la pagina sobre la que trabajar, adquirir los recursos web y, luego, organizar la información deseada. Como se ve en la figura \ref{fig:ej2}.
\newline
\newline
El primer paso consiste en mirar la estructuración de la web para seleccionar aquellos datos que queramos extraer, comprobando los recursos HTML, CSS y viendo si hace uso de JavaScript. Para realizar el segundo paso de adquisición de los recursos, el proceso empieza con los programas de web scraping enviando un request, ya sea mediante GET o POST, a la página web deseada. Una vez el request sea recibido y procesado por la web, esta enviará los recursos solicitados al programa. Después, pasaríamos al tercer paso, en el que analizaríamos los recursos obtenidos y los filtraríamos de tal forma que nos quedásemos únicamente con la información que nos sea necesaria. Finalmente almacenaríamos la información obtenida para su posterior análisis.
\newline
\begin{figure} [h!]
	\centering
	\includegraphics[width=0.7\textwidth]{fig/Web-Scraping-Adapted-from-Krotov-and-Tennyson-2018.png}
	\caption[Web Scraping (Krotov y Tennyson 2018)]{Web Scraping \footnotemark}
	\label{fig:ej2}
\end{figure}
\footnotetext{\url{https://www.researchgate.net/figure/Web-Scraping-Adapted-from-Krotov-and-Tennyson-2018_fig1_324907302}}

\subsection{Scrapy Framework}
Scrapy es un framework asíncrono para web crawling y web scraping gratuito y de código abierto. Por defecto proporciona todas las herramientas necesarias para realizar la tarea de extracción, procesado y estructurado de los datos adquiridos, ademas de dar la opción de extender funcionalidades en caso necesario, haciéndolo extremadamente versátil \cite{yang2019design}.
\newline
\newline
Pensado para navegar entre webs y extraer información de forma estructurada de ellas, Scrapy se usa para múltiples ámbitos, por ejemplo, el minado de datos o la monitorización y análisis de datos.\newline
\newline
A la hora de buscar herramientas para web scraping en Python, fueron tres la alternativas principales encontradas, BeautifulSoup, Selenium y Scrapy.
\newline
\newline
La primera es una librería de parseo de HTML y XML, que, aunque podría haber servido para cumplir el propósito del proyecto inicialmente, a la larga su sencillez hubiera sido más un problema que una ayuda.
\newline
\newline
La segunda por el contrario, no es una herramienta de web scraping como tal y, se centra en la automatización de la navegación web como entorno de pruebas. Debido a esto, no es una herramienta que haya usado para el web scraping, pero si que ha sido usada junto con Scrapy para navegar entre webs y sacar los datos de estas. Proporcionando la posibilidad de hacer uso de JavaScript sobre las páginas, pues Scrapy no dispone de renderizado JavaScript de forma nativa.
\newline
\newline
Scrapy fue elegido gracias a la flexibilidad y versatilidad que proporciona a la hora de trabajar, pudiendo crear proyectos sencillos en cuestión de minutos con las herramientas base proporcionadas o investigar como trabajar con estas herramientas y crear proyectos complejos que satisfagan tus necesidades de la manera que desees. Esto implica que la curva de aprendizaje de Scrapy sea mayor que la que puede tener BeautifulSoup, sobre todo al inicio, llegando a parecer abrumador. A su vez, la naturaleza de Scrapy le hace tener el mejor rendimiento de entre las tres \cite{glez2014web}.
\newline
\newline
Finalmente, cabe mencionar que Scrapy no dispone de rotación de IP, ni geolocalización como pueden tener las alternativas de pago, cosa que no es un impedimento para llevar a cabo el proyecto.\newline
\newline
Para satisfacer el segundo servicio necesario se requerirá de una base de datos.

\section{Base de Datos}
Debido a la naturaleza de los JSON obtenidos, se pueden trasladar fácilmente a tablas relacionales, por lo que se hará uso de una base de datos relacional.

\subsection{PostGreSQL}
PostGreSQL, es un sistema de gestión de bases de datos relacional de código abierto. PostgreSQL destaca por su sistema de gestión de bases de datos, su soporte para consultas complejas, transacciones ACID, integridad referencial y escalabilidad.\newline
\newline
Además, ofrece una amplia gama de tipos de datos, incluyendo geoespaciales y JSONB, almacenando JSONs de forma binaria (de ahí la B) para su fácil acceso.\newline
\newline
La comunidad activa detrás de PostgreSQL contribuye continuamente con mejoras y extensiones, lo que lo convierte en una solución versátil y confiable. Lo que la hace popular tanto para aplicaciones empresariales como para proyectos web.\newline
\newline
Entre las bases de datos con soporte oficial en Django se encuentran, PostgreSQL, MariaDB, MySQL, Oracle y SQLite. A excepción de Oracle, la única base de datos comercial, el resto son de código abierto, aunque puedan disponer de versiones de pago.\newline
\newline
A nivel de funcionalidades, todas ofrecen implementadas de una forma u otra una misma gama de estas. Es por eso que, la elección se hizo en base a la flexibilidad y robustez de estas. Lo que hace a PostGreSQL una opción interesante en proyectos que lleguen a requerir de datos geoespaciales y JSON.

\subsection{Arquitectura Preliminar}
Tras la selección de las herramientas a usar, la arquitectura queda de las manera mostrada en la figura \ref{fig:ej9}. Como herramienta de scraping, el framework Scrapy; como base de datos PostGreSQL y, como servicio API el framework Django.

\begin{figure} [H]
	\centering
	\includegraphics[width=0.3\textwidth]{fig/estructura_descartada.png}
	\caption[Idea original de la estructura de datos planteada]{Estructura descartada}
	\label{fig:ej9}
\end{figure}

El hecho de seleccionar estos Framework, impone la necesidad de usar como lenguaje de programación Python.\newline
\newline
Siendo su primera aparición en el año 1991 por manos de Guido van Rossum, Python es un lenguaje de programación de alto nivel interpretado que prima la legibilidad del código, siendo este a veces nominado como "seudocódigo ejecutable" \cite{dierbach2014python}. Python a su vez es un lenguaje multiplataforma, fuertemente tipado, dinámico y multiparadigma, pues soporta programación orientada a objetos, imperativa y funcional \cite{PyDoc} \cite{borges2014python}.\newline
\newline
