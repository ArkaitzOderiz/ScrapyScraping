\chapter[Tecnologías]{Tecnologías}
\label{Chap2}

\section{Debian}

\subsection{Historia}
Conforme la computación fue tomando terreno tanto en el ámbito comercial como en el escolar, no tener una forma de instalar y configurar los sistemas de una forma rápida sin tener que partir de cero y sin tener que compilar el software necesario manualmente se convirtió en un problema patente entre los usuarios.
\newline
\newline
En 1993, tras varios intento fallidos por distintas empresas de solucionar el problema, Ian Murdock, por aquel entonces estudiante de la Universidad Purdue, encontró la solución al problema basándose en el reciente proyecto de Linus Torvalds, el kernel Linux. tras el anuncio de Ian para crear un sistema operativo de forma descentralizada en paralelo como es el caso del kernel Linux, docenas de usuarios se unieron para formar el proyecto Debian Linux con la intención de crear un sistema operativo de gran calidad y mantenimiento, publicando en enero de 1994 la primera versión de Debian 0.91.

\subsection{Filosofía}
Debian no intenta seguir ni competir con los lideres del sector, por el contrario, desde sus inicios el proyecto se ha basado en una filosofía centrada en la robustez y estabilidad del sistema guiada por unos estrictos estándares de calidad, actualizándose conforme las necesidades de sus usuarios a la vez que promueve el software gratuito, lo que le ha ayudado a obtener fama entre los usuarios.
\newline
\newline
A su vez, debido al apoyo del software libre, hace uso de múltiples licencias de software como la Licencia Pública General GNU (GPL), licencias artísticas o del tipo BSD, lo cual ha llevado al desarrollo de las Directrices de Software Libre de Debian (DFSG) con el fin de definir la construcción del software libre.
\newpage
Definido por estas licencias, el software libre debe cumplir al menos las que están consideradas la cuatro libertades esenciales:
\begin{itemize}
	\item Ejecutar el programa como se desee, con cualquier propósito.
	\item Estudiar cómo funciona el programa, y cambiarlo para que haga lo que se desee.
	\item Redistribuir copias para ayudar a otros.
	\item Distribuir copias de sus versiones modificadas a terceros permitiendo ofrecer a toda la comunidad la oportunidad de beneficiarse de las modificaciones.
\end{itemize}

\subsection{Modelo de negocio}
Todo el modelo del proyecto Debian se basa en su Contrato Social respecto a la comunidad de software libre creado en 1997, sobre el cual se estipulan las directrices a seguir para crear y distribuir software libre.
\newline
\newline
No hay que confundir el termino libre, definido dentro de DFSG, con gratuito, pues un programa de software puede ser libre aunque sea de pago. Por el contrario, Debian es un proyecto libre y gratuito.
\newline
\newline
Debian se mantiene gracias a su comunidad, disponiendo más de mil desarrolladores y contribuidores por todo el mundo aportando al proyecto con su tiempo y conocimiento de forma gratuita. Esto, junto a la asistencia de múltiples empresas que dan soporte a Debian dentro del proyecto de socios y el sistema de donaciones usado para hardware, dominios, certificados criptográficos, conferencias, etc han ayudado en el éxito y crecimiento del proyecto dentro de su filosofía.

\subsection{Por qué Debian?}
El mercado está repleto de distintas posibles alternativas a Debian, ya sean de pago, Windows Server OS o Red Hat Enterprise Linux (RHEL), gratuitos, Ubuntu Server y Fedora Server, o incluso en la nube, Amazon Web Services (AWS), Google’s Cloud Platform.
\newline
\newline
Descartando todo sistema de pago, aunque las posibilidades se reducen aun hay múltiples opciones sobre las que elegir, pero son pocas las que ofrecen la misma usabilidad que Debian con sus más de 59000 paquetes en su versión estable o incluso puedes usar las versiones «en pruebas» o «inestable» en caso de querer probar las nuevas funciones antes de su lanzamiento oficial.
\newline
\newline
Aunque todos los sistemas basados en Linux disponen de las mismas características, siendo software libre y gratuito con soporte multi-usuario, multi-proceso y uso en tiempo real, Debian siendo uno de los sistemas más longevos del mercado a tomado fama entre la competencia por su seguridad y estabilidad, siendo la base para muchas de las distribuciones más populares contra las que compite, como Ubuntu, Knoppix, PureOS o Tails.
\newpage
Finalmente, la característica más distintiva de Debian frente a la competencia es su compatibilidad con un uso 24/7, pues otras alternativas gratuitas o no dan soporte a esta característica, Fedora Server o, necesitas disponer de una licencia de pago como es el caso de Ubuntu Server mediante Ubuntu Pro.
\newline
\newline
Una vez contado todo esto, parece lógica la elección de Debian como sistema operativo para usar en el servidor.

\newpage

\section{Django}
