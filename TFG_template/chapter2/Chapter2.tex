\chapter[Recogida de datos]{Recogida de datos}
\label{Chap2}

Código en R del ejemplo \ref{code:ej}:
	\begin{lstlisting}[language=R, caption=pie de código,label=code:ej]
		# Establecer las credenciales de la API [1]
		
		library(rsat)
		set_credentials("rsat.package", "UpnaSSG.2021")
		
		#Definir la region
		
		library(raster)
		dir.create( "./RSATprueba/countries" ,recursive = TRUE)
		spain<-getData('GADM', country= 'Spain', path= "./RSATprueba/countries", level=2)
		
	\end{lstlisting}


\begin{theorem}[Sum]\label{th:ex1}
$1+1=2$
\end{theorem}

\begin{definition}[Nice numbers]\label{def:ex1}
A number is \emph{nice} if it looks beautiful.
\end{definition}

\begin{theorem}[About $C^{1}(0,1)$]\label{th:ex2}
The set $C^{1}(0,1)$ is interesting.
\end{theorem}
Teoremaren erreferentzia \ref{th:ex2}

\begin{proof}\label{pf:ex1}
To prove it by contradiction try and assume that the statemenet is false,
proceed from there and at some point you will arrive to a contradiction.
\end{proof}



\begin{lemma}\label{lemma:ex1}
To prove it by contradiction try and assume that the statemenet is false,
proceed from there and at some point you will arrive to a contradiction.
\end{lemma}

Lemaren erreferentzia \ref{lemma:ex1}


\begin{equation}\label{eq:ex2}
  1 + e^{i \pi} = 0.
\end{equation}
Formularen erreferentzia \ref{eq:ex2}