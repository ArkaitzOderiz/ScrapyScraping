\chapter[Introducción]{Introducción}
\label{Chap1}

\section{Esto es una sección}
Aquí tenemos una imagen referenciada \ref{fig:ej2}. Una dirección \href{https://www.copernicus.eu/es}{Mi dirección}\footnotemark. 
			\begin{figure} [h!]
			\centering
			\includegraphics[width=0.7\textwidth]{fig/example-image.png}
			\caption[Nombre reducido para tabla de figuras]{Real caption}
			\label{fig:ej2}
		\end{figure}
		\footnotetext{https://www.copernicus.eu/es}

Para generar entradas en el índice de palabras\index{parabras}. SATA\index{SATA}.

\subsection{Esto es una subsección}
%un comentario
Una lista de parámetros: % con itemize se muestra una lista
	\begin{itemize}
		\item uno.
		\item dos.
		\item tres.
	\end{itemize}
	Una lista enumerada
	\begin{enumerate}
		\item  uno.
		\item  dos.
	\end{enumerate}
Vamos a citar \ldots \cite{lorenzi2011inpainting}
\begin{table}[H]\caption{Mi tabla de ejemplo}\label{tab:ej}
\begin{center}
	\begin{tabular}{c c c}
		Nombre & Medida & Otra cosa\\
		\hline
		10 & 10 & 4\\
	\end{tabular}
\end{center}
\end{table}