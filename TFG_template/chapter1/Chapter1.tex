\chapter[Introducción]{Introducción}
\label{Chap1}

Las Naciones Unidas definen el cambio climático como el conjunto de \textit{"cambios a largo plazo de las temperaturas y los patrones climáticos"} \cite{UNWeb}. Estos pueden ocurrir de forma natural cada cierto tiempo, ya sea debido a erupciones volcánicas, fluctuaciones de la radiación solar e, incluso, variaciones en la órbita terrestre, llegando a causar climatologías extremas como pueden ser las glaciaciones. Actualmente, a este proceso se le debe añadir como causa la actividad humana, alterando y agravando los efectos, siendo los principales causantes la contaminación, sobre-población y la deforestación.\newline
\newline
Partiendo desde principios de la revoluciona industrial a mediados del siglo XVIII, el efecto de la actividad humana como causa de estos cambios ha ido en aumento, siendo desde el siglo XIX hasta llegar a pleno siglo XXI la principal causa del cambio climático. Debido al uso extendido de combustibles fósiles tales como el carbón, petroleo y el gas, formando las principales fuentes de energía durante muchos años. Figura \ref{fig:ej16}.\newline
\newline
La quema de estos produce los llamados gases de efecto invernadero, principalmente dióxido de carbono y metano. Estos gases impiden la correcta liberación de la radiación irradiada por el suelo al ser calentado por el sol, absorbiendo parte de esta y liberándola nuevamente hacia la tierra, aumentando la temperatura de la superficie terrestre. Como puede verse en la figura \ref{fig:ej17}.

\begin{figure} [H]
	\centering
	\includegraphics[width=0.7\textwidth]{fig/global-co2-fossil-plus-land-use.png}
	\caption[Emisiones globales de CO2]{Emisiones globales de CO2 \footnotemark}
	\label{fig:ej16}
\end{figure}
\footnotetext{\url{https://ourworldindata.org/co2-emissions}}

\begin{figure} [H]
	\centering
	\includegraphics[width=0.7\textwidth]{fig/climate-change_greenhouse-effect_steps.png}
	\caption[Efecto invernadero]{Efecto invernadero \footnotemark}
	\label{fig:ej17}
\end{figure}
\footnotetext{\url{https://world101.cfr.org/global-era-issues/climate-change/greenhouse-effect}}

El aumento de emisiones de estos gases, a fomentado una crecida sustancial en la velocidad de elevación de la temperatura terrestre. Llegando a medirse un aumento regional de, entre 0.8\textdegree C hasta 2\textdegree C en altas latitudes, estimando un aumento en la media global de 1.1\textdegree C en comparación con finales del siglo XIX \cite{ruddiman2003anthropogenic}.\newline
\newline
Década tras década las temperaturas han ido aumentando, llegando a un punto en el que prácticamente anualmente se baten récords de temperatura máximas por todo el globo. Siendo globalmente el verano de 2023 el más caluroso registrado desde 1850. Figura \ref{fig:ej18} \cite{NCEIWeb}.

\begin{figure} [H]
	\centering
	\includegraphics[width=0.7\textwidth]{fig/ClimateDashboard-global-surface-temperature-graph-20230118-1400px.png}
	\caption[Media de aumento de temperatura global entre 1901-2000]{Media de aumento de temperatura global entre 1901-2000 \footnotemark}
	\label{fig:ej18}
\end{figure}
\footnotetext{\url{https://www.climate.gov/news-features/understanding-climate/climate-change-global-temperature}}

Son múltiples los efectos adversos causado por el aumento de la temperatura, entre ellos, la subida del nivel marino, reduciendo e inundando zonas costeras, la desertificación de zonas actualmente áridas y la alteración del comportamiento de especies tanto animales como vegetales, llegando a suponer una reducción sobre la población en multitud de especies \cite{arnell2019global} \cite{new2011four}.\newline
\newline
Es por eso que múltiples países son ya los que optan por comprometerse a alcanzar una cota de emisiones cero para 2050, tratando de reducir las emisiones globales a la mitad cara 2030, con el fin de mantener el aumento de la temperatura media por debajo de 1.5\textdegree C, estimando esta como un punto reflexivo a la hora de controlar el impacto climático, con el objetivo de mantener un clima habitable.\newline
\newline
Pero el problema no reside únicamente en el aumento de la temperatura, como da a entender la definición proporcionada por la ONU, llegando a influenciar sobre los patrones climáticos. Empezando por las estaciones, se ha llegado a observar fluctuaciones en estas, anticipando de la llegada de la primavera, prolongando el verano, retrasando el otoño y reduciendo en la temporada invernal \cite{sparks2002observed}.\newline
\newline
Sumado a esto, son cada vez mas los efectos visibles causados sobre los fenómenos naturales. Es tal el impacto que, ya son más de veinte las anomalías climáticas de una importancia significante remarcadas por el centro nacional de información ambiental (NCEI) en 2022 \cite{NCEIWeb}. Entre ellas, sequías debido al aumento de las temperaturas, llegando a fomentar la aparición de incendios y, fuertes lluvias, marcadas por el aumento de tormentas, huracanes y tifones durante las estaciones lluviosas, causando múltiples inundaciones.\newline
\newline
El cambio climático agrava los efectos de los fenómenos naturales, reduciendo su periodo de retorno (periodo estimado de años para la aparición de un fenómeno climático), causante de esta meteorología extrema. Zonas planetarias enteras se ven azotadas por graves sequías y tormentas, modificando la disponibilidad del agua sobre la superficie terrestre, cosa que, afecta severamente a la agricultura. La precipitación acumulada se mantiene debajo de la media globalmente, mientras que la frecuencia de aparición de precipitaciones anómalas va en aumento, figura \ref{fig:ej19}, fomentadas por una mayor evaporación de agua a causa del aumento de la temperatura, agravando la frecuencia de tormentas de magnitudes extremas. 

\begin{figure} [H]
	\centering
	\begin{subfigure}{.5\textwidth}
		\centering
		\includegraphics[width=.9\linewidth]{fig/map-prcp-202201-202212.png}
		\caption{Precipitaciones Anómalas}
		\label{fig:sub1}
	\end{subfigure}%
	\begin{subfigure}{.5\textwidth}
		\centering
		\includegraphics[width=.9\linewidth]{fig/map-prcp-percent-202201-202212.png}
		\caption{Media de precipitaciones}
		\label{fig:sub2}
	\end{subfigure}
	\caption{Precipitaciones 2022}
	\label{fig:ej19}
\end{figure}

Estos efectos no tienen por que darse independientemente, sin ir más lejos, en 2022, España batió el récord como tercer año más seco registrado, solo superado por los años 2005 y 2017, más tarde, ese mismo año en diciembre, fuertes lluvias causaron múltiples inundaciones provocando daños materiales sobre viviendas y carreteras \cite{NCEIWebPreci}.\newline
\newline
Las inundaciones causadas por lluvias torrenciales son una de las principales causantes de daños materiales a nivel global \cite{wasko2021incorporating}, estimando una perdida anual de 800 millones de euros tan solo en España \cite{Miteco}. Es por eso que, la monitorización y predicción climática son primordiales a la hora de procurar reducir su efecto a nivel económico como humano.\newline
\newline
Los últimos 15-20 años la Unión Europea ha realizado un gran esfuerzo en promover políticas de datos abiertos en realización a la información generada y monitorizada por los estados miembros. El objetivo de esta política es acercar estos datos a la población para tener un mayor control territorial y medio ambiental. El gobierno de Navarra y de España contribuyen a la oferta de datos abiertos publicando mediante diferentes organismos como el geoportal (\url{https://geoportal.navarra.es/es/idena}) en Navarra o los datos o el sistema automático de información publicada por las diferentes confederaciones hidrográficas en España.\newline
\newline
A fin de obtener datos tanto fluviales como pluviométricos, disponiendo de los presentes en las siguientes agencias y organismos.\newline
\newline
La Confederación Hidrográfica del Cantábrico (\url{https://www.chcantabrico.es/}) es un organismo autónomo adscrito al Ministerio para la Transición Ecológica y el Reto Demográfico. Es responsable de la gestión de las cuencas hidrográficas de los ríos que vierten al mar Cantábrico, ejerciendo sobre las comunidades autónomas de: Principado de Asturias, Cantabria, Castilla y León, Galicia, País Vasco y Comunidad Foral de Navarra.\newline
\newline
La Agencia Estatal de Meteorología (AEMET) (\url{https://www.aemet.es/es/portada}), también adscrita al Ministerio para la Transición Ecológica y el Reto Demográfico, tiene como objetivo el desarrollo, implantación, y prestación de los servicios meteorológicos del Estado, apoyando al ejercicio de otras políticas públicas y actividades privadas, contribuyendo a la seguridad de personas y bienes, y al bienestar y desarrollo sostenible de la sociedad.\newline
\newline
Meteo Navarra (\url{http://meteo.navarra.es/}), es la agencia encargada de la  monitorización de la meteorología y climatología de Navarra, trabaja junto al gobierno de Navarra, el Ministerio de Agricultura, Pesca y Alimentación (MAPA), el Instituto Navarro de Tecnologías e Infraestructuras Agroalimentarias (INTIA), la Agencia Estatal de Meteorología (AEMET) y la Universidad Pública de Navarra (UPNA).\newline
\newline
El agua en Navarra (\url{http://www.navarra.es/home_es/Temas/Medio+Ambiente/Agua/}), otra plataforma del gobierno de Navarra, esta vez centrada en los recursos hídricos disponibles en Navarra.\newline
\newline
Con el fin de posibilitar la creación un modelo de predicción de inundaciones sobre los ríos en Navarra, el objetivo de este trabajo es crear una plataforma habilitada en la obtención y centralizando de la información proveniente de estas diferentes fuentes, figura \ref{fig:ej31}. Mediante esta plataforma se pretende disponer de un conglomerado de datos sobre los que trabajar con el propósito de prevenir sobre inundaciones a la población navarra por medio de un sistema de notificaciones automáticas.\newline
\newline

\begin{figure} [H]
	\centering
	\includegraphics[width=0.4\textwidth]{fig/estructura_general.png}
	\caption[Estructura de datos planteada]{Estructura de datos planteada}
	\label{fig:ej31}
\end{figure}

Este trabajo pretende obtener la mayor cantidad de datos posibles con el fin de predecir dichas inundaciones causadas por por las crecidas de los ríos, de modo que las administraciones y la población puedan actuar con tiempo y reducir el efecto devastador y económico producido por estos efectos. Como prueba de concepto trabajaremos a nivel regional en España. En concreto, la Comunidad Foral de Navarra.\newline
\newline
El resto de las memoria se divide de la siguiente manera, la Sección 2 presenta las fuentes de obtención de datos y los datos necesarios para realizar el proceso de predicción; la Sección 3 presenta las tecnologías necesarias para realizar la arquitectura planteada; en la Sección 4 se ahonda en el diseño de la arquitectura, explicando los componentes de esta y como preparar el entorno para realizarla; la Sección 5 presenta y explica el código implementado con el fin de crear la plataforma. Finalmente, en la Sección 6 se presentan las conclusiones y el trabajo futuro.