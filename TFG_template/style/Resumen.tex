\phantomsection
\chapter*{Resumen}
Los últimos 15-20 años la Unión Europea ha realizado un gran esfuerzo en promover políticas de datos abiertos en realización a la información generada y monitorizada por los estados miembros. El objetivo de esta política es acercar estos datos a la población para tener un mayor control territorial y medio ambiental. El gobierno de Navarra y de España contribuyen a la oferta de datos abiertos publicando mediante diferentes organismos como el geoportal (\url{https://geoportal.navarra.es/es/idena}) en Navarra o los datos o el sistema automático de información publicada por las diferentes confederaciones hidrográficas en España.\newline
\newline
El objetivo de este trabajo es centralizar la información proveniente de diferentes fuentes para la predicción y el aviso de inundaciones mediante los datos pluviométricos de los ríos de Navarra. para ello se necesitará de una plataforma habilitada en técnicas de scraping sobre los datos ya publicados en múltiples plataformas web, los cuales serás guardados en una base de datos.
\newline
\newline
\newline
\newline
\textbf{{\LARGE Palabras clave}}
\newline
\newline
Herramientas de Scraping, Framework, HTML, JSON, web scraping, GET request, POST request, API, arquitectura, plataforma, BBDD, Entornos virtuales, Spider, Runner, Executer